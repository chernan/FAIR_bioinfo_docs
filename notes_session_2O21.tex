

\begin{frame}{git}
\begin{itemize}
  \item rachat github par microsoft : à ne mentionner qu'1 fois plutôt que plusieurs
  \item d13: commencer par un git version pour voir si git est présent en local et sinon, ajouter un déplacement "cd" (-w) dans le docker pour ne pas faire un git depuis la racine
  \item d17: ajouter cd .. avant le clone
\end{itemize}
\end{frame}
\begin{frame}{github}
\begin{itemize}
  \item ajouter comment créer un dépôt github
  \item d28: branche principale a changé de nom : master / main
  \item d50/d263 : changer le nom du fork
\end{itemize}
\end{frame}
\begin{frame}{conda}
\begin{itemize}
  \item d35: supprimer le "env" au create
  \item d35: install bowtie2 si erreur libppt.so => deactivate, conda clean --all, activate
  \item ajouter capture écran du site conda hub avant le 1er install pour voir d'où provient les outils/chanels
\end{itemize}
\end{frame}
\begin{frame}{smk}
\begin{itemize}
    \item préparer sur une clef usb l'image avec smk-minimal, fastqc et multiqc car bcp trop long avec la connection du iscpif  (institut des systemes complexes paris ile de france) ! ou alors faire une pause Kfé pdt ce tps !
    \item d22: "input" est après le répertoire
    \d??: directory("multiqc_data") => provoque un warnng 
\end{itemize}
\end{frame}
\begin{frame}{litterate prog}
\begin{itemize}
    \item ajouter url article
\end{itemize}
\end{frame}
