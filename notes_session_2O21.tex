
\begin{frame}{Notes générales}
\begin{itemize}
  \item Exporter les images Docker avant pour éviter les téléchargements
  \item penser à laisser des passages entre les tables pour faciliter le déplacement des helpers
  \item Ce(A)C(nrs)I(NRIA)L(ogiciel)L(ibre)
  \item comparaison github/gitlab
  \item vérifier les temps nécessaires à chaque partie (ex: sharing 1/2h ok, cluster plutôt 2h, RMarkdown 1h un peu court avec TP mais probablement que 1h30 serait long...)
  \item ajouter un lien vers le Community IFB, vers d'autres formations pour aller plus loin
  \item re-centrer sur des notebooks?
\end{itemize}
\end{frame}

\begin{frame}{git}
\begin{itemize}
  \item rachat github par microsoft : à ne mentionner qu'1 fois plutôt que plusieurs (est aussi dans la partie GitHub)
  \item d13: commencer par un git version pour voir si git est présent en local et sinon, ajouter un déplacement "cd" (-w) dans le docker pour ne pas faire un git depuis la racine
  \item d17: ajouter cd .. avant le clone
\end{itemize}
\end{frame}

\begin{frame}{github}
\begin{itemize}
  \item ajouter comment créer un dépôt github et la discussion sur les licences
  \item ajouter des slides pour présenter les onglets dans l'interface
  \item d28: branche principale a changé de nom : master / main
  \item d50/d263 : changer le nom du fork
\end{itemize}
\end{frame}

\begin{frame}{conda}
\begin{itemize}
  \item d35: supprimer le "env" au create
  \item d35: install bowtie2 si erreur libppt.so => deactivate, conda clean --all, activate
  \item ajouter capture écran du site conda hub avant le 1er install pour voir d'où provient les outils/chanels
\end{itemize}
\end{frame}

\begin{frame}{docker}
\begin{itemize}
    \item mettre à jour le dockerfile dans le tout dernier exercice
    \item chenger le docker start (enlever les paramères et /bin/bash)
    \item les machines linux ont besoin de lancer docker avec sudo ?!
    \item : ajouter cc sur pieds de page
    \item ajouter docker volume? docker system prune ?
\end{itemize}
\end{frame}

\begin{frame}{smk}
\begin{itemize}
    \item préparer sur une clef usb l'image avec smk-minimal, fastqc et multiqc car bcp trop long avec la connection du iscpif  (institut des systemes complexes paris ile de france) ! ou alors faire une pause Kfé pdt ce tps !
    \item d22: "input" est après le répertoire
    \item d??: directory("multiqc\_data") => provoque un warning 
    \item suggestion Gildas : inclure https://workflowhub.eu/ et le projet RO-Crate.
\end{itemize}
\end{frame}

\begin{frame}{litterate prog}
\begin{itemize}
    \item ajouter url article
    \item exporter l'image rstudio avant
    \item essayer d'alléger l'image...
\end{itemize}
\end{frame}

