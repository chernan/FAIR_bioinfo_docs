\documentclass{beamer}
\usepackage[utf8]{inputenc}
\usepackage{color}
\usepackage{utopia} %font utopia imported
\usetheme{Madrid}
\usecolortheme{default}
%------------------------------------------------------------
%This block of code defines the information to appear in the
%Title page
\title[FAIR{\_}Bioinfo] %optional
{FAIR{\_}bioinfo for bioinformaticians}

\subtitle{Introduction to the tools of reproducibility in bioinformatics}

\author[Céline, Claire] % (optional)
{C.~Hernandez\inst{1} \and T.~Denecker\inst{1} \and C.~Toffano-Nioche\inst{1}}

\institute[I2BC] % (optional)
{
  \inst{1}%
  Institute for Integrative Biology of the Cell (I2BC)\\
  UMR 9198, Université Paris-Sud, CNRS, CEA\\
  91190 - Gif-sur-Yvette, France
%  \and
%  \inst{2}%
%  Faculty of Chemistry\\
%  Very Famous University
}

\date[IFB 2020] % (optional)
{Sept. 2020}

\logo{\includegraphics[height=1cm]{shared/logo-ifb.jpg}}


%End of title page configuration block
%------------------------------------------------------------
\begin{document}
 \newcommand{\logoAws}{\protect\includegraphics[height=1.7ex,keepaspectratio]{shared/logo-aws.png}}
\newcommand{\logoCmake}{\protect\includegraphics[height=1.7ex,keepaspectratio]{shared/logo-cmake.png}}
\newcommand{\logoCNRS}{\protect\includegraphics[height=1.7ex,keepaspectratio]{shared/logo-CNRS.png}}
\newcommand{\logoConda}{\protect\includegraphics[height=1.7ex,keepaspectratio]{shared/logo-conda.png}}
\newcommand{\logoDockerPaysage}{\protect\includegraphics[height=1.7ex,keepaspectratio]{shared/logo-docker-paysage.png}}
\newcommand{\logoDockerPortrait}{\protect\includegraphics[height=1.7ex,keepaspectratio]{shared/logo-docker-portrait.png}}
\newcommand{\logoDockerHub}{\protect\includegraphics[height=1.7ex,keepaspectratio]{shared/logo-dockerHub.png}}
\newcommand{\logoGit}{\protect\includegraphics[height=1.7ex,keepaspectratio]{shared/logo-git.png}}
\newcommand{\logoGithub}{\protect\includegraphics[height=1.7ex,keepaspectratio]{shared/logo-github.png}}
\newcommand{\logoGoogleCloud}{\protect\includegraphics[height=1.7ex,keepaspectratio]{shared/logo-googleClooud.png}}
\newcommand{\logoIdeuxBC}{\protect\includegraphics[height=1.7ex,keepaspectratio]{shared/logo-i2bc.png}}
\newcommand{\logoIFB}{\protect\includegraphics[height=1.7ex,keepaspectratio]{shared/logo-ifb.jpg}}
\newcommand{\logoJupyter}{\protect\includegraphics[height=1.7ex,keepaspectratio]{shared/logo-jupyter.png}}
\newcommand{\logoMicrosoftAzure}{\protect\includegraphics[height=1.7ex,keepaspectratio]{shared/logo-microsoftAzure.png}}
\newcommand{\logoNextflow}{\protect\includegraphics[height=1.7ex,keepaspectratio]{shared/logo-nextflow.png}}
\newcommand{\logoRmarkdown}{\protect\includegraphics[height=1.7ex,keepaspectratio]{shared/logo-Rmarkdown.png}}
\newcommand{\logoShiny}{\protect\includegraphics[height=1.7ex,keepaspectratio]{shared/logo-shiny.png}}
\newcommand{\logoSingularity}{\protect\includegraphics[height=1.7ex,keepaspectratio]{shared/logo-singularity.png}}
\newcommand{\logoSlurm}{\protect\includegraphics[height=1.7ex,keepaspectratio]{shared/logo-slurm.png}}
\newcommand{\logoSnakemake}{\protect\includegraphics[height=1.7ex,keepaspectratio]{shared/logo-snakemake.png}}
\newcommand{\logoVirtualbox}{\protect\includegraphics[height=1.7ex,keepaspectratio]{shared/logo-virtualbox.png}}
\newcommand{\logoZenodo}{\protect\includegraphics[height=1.7ex,keepaspectratio]{shared/logo-zenodo.png}}

%-------------------------------------------
\begin{frame}
%-------------------------------------------
    \titlepage
\end{frame}
%-------------------------------------------
\begin{frame}{Objectifs pédagogiques, public ciblé }
%-------------------------------------------
\begin{description}
  \item[Public cible :] ayant des bases en bio-informatique
  \item[Objectifs :] introduction à des outils permettant d'améliorer la reproductibilité des analyses bioinformatiques
\end{description}


{\bf \textcolor{blue}{Cible 2 savoir-faire métier :}}
\begin{itemize}
        \item traitement automatisé de données brutes
        \item analyse des données traitées
    \end{itemize} 
{\bf \textcolor{blue}{Pédagogie en 2 temps :}}
    \begin{itemize}
        \item introduction des fonctionnalités (courtes présentations)
        \item illustration par des travaux pratiques (RNA-seq DEG)
    \end{itemize}
\end{frame}
%-------------------------------------------
\begin{frame}{Programme}
%-------------------------------------------
\begin{center}
Sur deux jours, 6 sessions d'environ 1h30.
\end{center}
\begin{center}
\begin{tabular}{| r | l |}
\hline
    Introduction to reproducibility & \\
    \hline \hline
    Encapsulate tasks execution & docker\\
    \hline
    Manage executions for parallelization & snakemake\\ 
    \hline \hline
    Manage a software environment, versioning & conda, git \\
    \hline
    Traceability with notebooks & jupyter \\
    \hline \hline
    Share and deploy & github, zenodo \\
\hline
\end{tabular}
\end{center}
\end{frame}
%-------------------------------------------
\begin{frame}{Sessions pratiques 1/2}
%-------------------------------------------
\begin{center}
{\bf \textcolor{blue}{1. Introduction to reproducibility}}
\begin{enumerate}
    \item pas de session pratique
\end{enumerate}
\end{center}
\begin{center}
{\bf \textcolor{blue}{2. Encapsulate tasks execution}}
\begin{enumerate}
    \item interactions avec une image docker (samtools flagstat)
    \item lancements de containers depuis un script bash
    \item utilisation/démo de singularity sur le cluster de l'IFB ?
\end{enumerate}
{\bf \textcolor{blue}{3. Workflow management}}
\begin{enumerate}
    \item snakemake simple pour un outil du pipeline (samtools flagstat)
    \item lancement d'un workflow plus complexe (fourni) en local
    \item idem mais sur le cluster IFB ?
\end{enumerate}
\end{center}
\end{frame}
%-------------------------------------------
\begin{frame}{Sessions pratiques 2/2}
%-------------------------------------------
\begin{center}
{\bf \textcolor{blue}{4. Manage a software environment}}
\begin{enumerate}
    \item conda : manipulation de base de conda
    \item conda : installation de samtools
    \item git : installation de git avec conda
    \item git : manipulation de git
\end{enumerate}
{\bf \textcolor{blue}{5. Traceability with notebooks}}
\begin{enumerate}
%    \setcounter{enumi}{2}
    \item utilisation de RStudio (install avec conda ou instance IFB ?)
%    \setcounter{enumi}{0}
    \item créer/exécuter le notebook de l'analyse DESeq2 (table DE, vocano-plot)
    \item {\it (si temps : création/démo d'une mini-appli Rshiny)}
\end{enumerate}
{\bf \textcolor{blue}{6. Share and deploy}}
\begin{enumerate}
    \item démo? DOI, Zenodo, Release, Licence, basculer sur GitHub
    \item partie pratique à définir
\end{enumerate}
\end{center}
\end{frame}
%-------------------------------------------
\begin{frame}{Matériels pédagogiques}
%-------------------------------------------
Accès aux données (à terme) : 
\begin{center}
%    \logoCNRS : dépôt sur mycore (CNRS) - \logoGithub : dépôt sur GitHub
    \logoIdeuxBC : dépôt sur I2BC - \logoGithub : dépôt sur GitHub
\end{center}
\begin{itemize}
  \item raw data ($\sim$750 Mo \logoIdeuxBC): 6 RNAseq ($\sim 6*20$ Mo), génome ($13$ Mo), annotations ($9$ Mo)
  \item processed data \logoIdeuxBC: bam, counts table
  \item scripts \logoGithub: pipeline (script bash, snakemake), notebooks
  \item slides (\LaTeX beamer/PDF \logoGithub,\logoIdeuxBC)
\end{itemize}
\end{frame}
\end{document}