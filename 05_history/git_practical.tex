%-------------------------------------------
\begin{frame}[containsverbatim]
\frametitle{\raisebox{-1.5ex}{\includegraphics[height=0.8cm]{shared/logo-git.png}} Objective}
%-------------------------------------------
\begin{exampleblock}{1st exercise}
\begin{enumerate}
    \item access and configure git
    \item initialize a git repository
    \item create files in this repository
    \item use the basic git commands for tracking files changes (status, add, commit)
\end{enumerate}
\end{exampleblock}
\begin{exampleblock}{2nd exercise}
\begin{enumerate}
\setcounter{enumi}{4}
    \item copy another repository from github (clone)
    \item use branching (branch) and merging (merge) to manage code changes
\end{enumerate}
\end{exampleblock}
\end{frame}
%-------------------------------------------
\begin{frame}[containsverbatim]
\frametitle{\raisebox{-1.5ex}{\includegraphics[height=0.8cm]{shared/logo-git.png}}  setup: objectives 1 $\&$ 2}
%-------------------------------------------
\begin{exampleblock}{Git access by doker}
\begin{lstlisting}[language=python]
docker run -i -t -v ${PWD}:/data continuumio/miniconda3
\end{lstlisting}
\end{exampleblock}
\begin{exampleblock}{Git configuration}
Global configuration (checking \verb|user.name| with: \verb|git config --list|):
\begin{lstlisting}[language=python]
git config --global user.name 'Your Name'
git config --global user.email 'Your Email'
\end{lstlisting}
\end{exampleblock}
\begin{exampleblock}{Git repository initialization}
On a new dedicated folder run: 
\begin{lstlisting}[language=python]
git init # observe the .git folder (ls -la)
git status # find the current branch, "nothing to commit"
\end{lstlisting}
\end{exampleblock}
\end{frame}
%-------------------------------------------
\begin{frame}[containsverbatim]
\frametitle{\raisebox{-1.5ex}{\includegraphics[height=0.8cm]{shared/logo-git.png}} adding files: objective 3}
%-------------------------------------------
\begin{exampleblock}{create 2 files, check their git status: obj}
\begin{lstlisting}[language=python]
for i in 1 2 ; do echo "file"${i}" text" > file${i}.txt ; done
git status # observe list of untracked files
\end{lstlisting}
\end{exampleblock}
\begin{exampleblock}{add file1 to stating area}
\begin{lstlisting}[language=python]
git add file1.txt
git status # observe the changing status of file1: untracked => staged
\end{lstlisting}
\end{exampleblock}
\begin{exampleblock}{change file1 text}
\begin{lstlisting}[language=python]
sed 's/text/text change/' file1.txt > tmp ; mv tmp file1.txt
git status # observe the 3 states, why file1 appears in "to be commited" and also in "not staged for commit"?
\end{lstlisting}
\end{exampleblock}
\end{frame}
%-------------------------------------------
\begin{frame}[containsverbatim]
\frametitle{\raisebox{-1.5ex}{\includegraphics[height=0.8cm]{shared/logo-git.png}} commit: objective 4}
%-------------------------------------------
\begin{exampleblock}{stage all files}
\begin{lstlisting}[language=python]
git add file1.txt file2.txt # all files
git status
\end{lstlisting}
\end{exampleblock}
\begin{exampleblock}{commit}
\begin{lstlisting}[language=python]
git commit -m "1st commit + file1 change" # always add a message, use present time to explain the change
git status # all ok
\end{lstlisting}
\end{exampleblock}
\end{frame}
%-------------------------------------------
\begin{frame}{\raisebox{-1.5ex}{\includegraphics[height=0.8cm]{shared/logo-git.png}}}
%-------------------------------------------
\begin{exampleblock}{}%summary
     So far, we have initiated a new project whose code is versioned by git: we have created files and all their successive changes were saved thanks to git.
\end{exampleblock}
\begin{exampleblock}{}%next step
    We will now create a 2nd project by copying an already existing one.\\ We're going to bring this project from an online git project site, \textit{e.g.} github.
\end{exampleblock}
\end{frame}
%-------------------------------------------
\begin{frame}[containsverbatim]
\frametitle{\raisebox{-1.5ex}{\includegraphics[height=0.8cm]{shared/logo-git.png}} cloning: objective 5}
%-------------------------------------------
\begin{exampleblock}{copy of a project: clone}
To download a project from github, we use the git \verb|clone| command:
\begin{lstlisting}
git clone https://github.com/clairetn/FAIR_bioinfo_github.git
\end{lstlisting}
\end{exampleblock}
\begin{exampleblock}{observe result}
\begin{itemize}
    \item a new folder has been created (check with the shell \verb|ls| command)
    \item its name is directly deduced from the url used
    \item this \verb|FAIR_bioinfo_github| folder contains a \verb|.git| repository and also a \verb|README.md| file (see with \verb|ls -la FAIR_bioinfo_github/|)
    \item it is a minimal project!
\end{itemize}
\end{exampleblock}
\end{frame}
%-------------------------------------------
\begin{frame}[containsverbatim]
\frametitle{\raisebox{-1.5ex}{\includegraphics[height=0.8cm]{shared/logo-git.png}} branching: objective 6}
%-------------------------------------------
\begin{exampleblock}{}
We plan to change the README file by adding our firstname at the authors list. With a git versioning system, a good practice is to create a branch to reserve the initial code until we validate our change.
\end{exampleblock}
\begin{exampleblock}{create a branch nammed "branch1"}
\begin{lstlisting}[language=python]
cd FAIR_bioinfo_github
git branch branch1
\end{lstlisting}
\end{exampleblock}
\begin{exampleblock}{list all branches}
\begin{lstlisting}[language=python]
git branch # find the star
\end{lstlisting}
\end{exampleblock}
\end{frame}
%-------------------------------------------
\begin{frame}[containsverbatim]
\frametitle{\raisebox{-1ex}{\includegraphics[height=0.8cm]{shared/logo-git.png}} branching: objective 6}
%-------------------------------------------
\begin{exampleblock}{go into the new "branch1"}
\begin{lstlisting}[language=python]
git checkout branch1
git branch # find the star
git status # find the branch
\end{lstlisting}
% ls -la # copy of the master branch (see date)
\end{exampleblock}
\begin{exampleblock}{work into branch: change a file and keep change}
Edit the \verb|README.md| file and add your firstname to the "Authors list"
\begin{lstlisting}[language=python]
git status # file README.md is modified
git add README.md ; git commit -m "add my firstname in branch1"
\end{lstlisting}
\end{exampleblock}
\begin{exampleblock}{return to master branch}
\begin{lstlisting}[language=python]
git checkout master
more README.md # Is README.md modified or initial version?
\end{lstlisting}
\end{exampleblock}
\end{frame}
%-------------------------------------------
\begin{frame}[containsverbatim]
\frametitle{\raisebox{-1.5ex}{\includegraphics[height=0.8cm]{shared/logo-git.png}} merging branch: objective 6}
%-------------------------------------------
\begin{exampleblock}{}
We have check that our change is valid, so we now plan to move it into the \verb|master| branch.
\end{exampleblock}
\begin{exampleblock}{merge branch, then delete branch}
\begin{lstlisting}[language=python]
git merge branch1
more README.md # what README.md version?
git branch -d branch1 # -d for delete
\end{lstlisting}
\end{exampleblock}
\end{frame}