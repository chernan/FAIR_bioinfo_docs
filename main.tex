\documentclass{beamer}
\usepackage[utf8]{inputenc}
\usepackage{utopia} %font utopia imported
%-------------------------------------------------
% slide appearance
\usetheme{Boadilla}
\usecolortheme{default}

%-------------------------------------------------
% listing
\usepackage{listings}
\usepackage{xcolor}
\definecolor{codegreen}{rgb}{0,0.6,0}
\definecolor{codegray}{rgb}{0.5,0.5,0.5}
\definecolor{codepurple}{rgb}{0.58,0,0.82}
\definecolor{backcolour}{rgb}{0.95,0.95,0.92}
\lstdefinestyle{mystyle}{
    backgroundcolor=\color{backcolour},   
    commentstyle=\color{codegreen},
    keywordstyle=\color{magenta},
    numberstyle=\tiny\color{codegray},
    stringstyle=\color{codepurple},
    basicstyle=\ttfamily\footnotesize,
    breakatwhitespace=false,         
    breaklines=true,                 
    captionpos=b,                    
    keepspaces=true,                 
    numbers=left,                    
    numbersep=5pt,                  
    showspaces=false,                
    showstringspaces=false,
    showtabs=false,                  
    tabsize=2
}
\lstset{style=mystyle}
%--------------------------------------------------

% aide LaTeX : http://l.roussarie.free.fr/IMG/pdf/ltx101-4-2.pdf

\newcommand\FAIRB{FAIR{\_}bioinfo}


%------------------------------------------------------------
%This block of code defines the information to appear in the
%Title page
\title[FAIR{\_}Bioinfo] %optional
{FAIR{\_}bioinfo for bioinformaticians}

\subtitle{Introduction to the tools of reproducibility in bioinformatics}

\author[Céline, Thomas, Claire] % (optional)
{C.~Hernandez\inst{1} \and T.~Denecker\inst{1} \and J.Sellier\inst{2}\and C.~Toffano-Nioche\inst{1}}

\institute[I2BC-IFB] % (optional)
{
  \inst{1}%https://fr.overleaf.com/project/5e749e3c458ed8000166a12b
  Institute for Integrative Biology of the Cell (I2BC)\\
  UMR 9198, Université Paris-Sud, CNRS, CEA\\
  91190 - Gif-sur-Yvette, France
\and
  \inst{2}%
  Institut Français de Bioinformatique\\
  \alert{à compléter}
}

\date[IFB 2020] % (optional)
{Sept. 2020}
\logo{\includegraphics[height=0.5cm]{shared/logo-ifb.jpg}}

%End of title page configuration block
%------------------------------------------------------------
\begin{document}

%------------------------------------------------------------
%The next block of commands puts the table of contents at the 
%beginning of each section and highlights the current section:

%------------------------------------------------------------
\AtBeginSection[]
{
  \begin{frame}{Table of contents}
    \begin{columns}[t]
        \begin{column}{.4\textwidth}
            \tableofcontents[currentsection, hideothersubsections, sections={1-3}]
        \end{column}
        \begin{column}{.4\textwidth}
            \tableofcontents[currentsection, hideothersubsections, sections={4-9}]
        \end{column}
    \end{columns}
  \end{frame}
}
%------------------------------------------------------------


%------------------------------------------------------------
%logos definition
 \newcommand{\logoAws}{\protect\includegraphics[height=1.7ex,keepaspectratio]{shared/logo-aws.png}}
\newcommand{\logoCmake}{\protect\includegraphics[height=1.7ex,keepaspectratio]{shared/logo-cmake.png}}
\newcommand{\logoCNRS}{\protect\includegraphics[height=1.7ex,keepaspectratio]{shared/logo-CNRS.png}}
\newcommand{\logoConda}{\protect\includegraphics[height=1.7ex,keepaspectratio]{shared/logo-conda.png}}
\newcommand{\logoDockerPaysage}{\protect\includegraphics[height=1.7ex,keepaspectratio]{shared/logo-docker-paysage.png}}
\newcommand{\logoDockerPortrait}{\protect\includegraphics[height=1.7ex,keepaspectratio]{shared/logo-docker-portrait.png}}
\newcommand{\logoDockerHub}{\protect\includegraphics[height=1.7ex,keepaspectratio]{shared/logo-dockerHub.png}}
\newcommand{\logoGit}{\protect\includegraphics[height=1.7ex,keepaspectratio]{shared/logo-git.png}}
\newcommand{\logoGithub}{\protect\includegraphics[height=1.7ex,keepaspectratio]{shared/logo-github.png}}
\newcommand{\logoGoogleCloud}{\protect\includegraphics[height=1.7ex,keepaspectratio]{shared/logo-googleClooud.png}}
\newcommand{\logoIdeuxBC}{\protect\includegraphics[height=1.7ex,keepaspectratio]{shared/logo-i2bc.png}}
\newcommand{\logoIFB}{\protect\includegraphics[height=1.7ex,keepaspectratio]{shared/logo-ifb.jpg}}
\newcommand{\logoJupyter}{\protect\includegraphics[height=1.7ex,keepaspectratio]{shared/logo-jupyter.png}}
\newcommand{\logoMicrosoftAzure}{\protect\includegraphics[height=1.7ex,keepaspectratio]{shared/logo-microsoftAzure.png}}
\newcommand{\logoNextflow}{\protect\includegraphics[height=1.7ex,keepaspectratio]{shared/logo-nextflow.png}}
\newcommand{\logoRmarkdown}{\protect\includegraphics[height=1.7ex,keepaspectratio]{shared/logo-Rmarkdown.png}}
\newcommand{\logoShiny}{\protect\includegraphics[height=1.7ex,keepaspectratio]{shared/logo-shiny.png}}
\newcommand{\logoSingularity}{\protect\includegraphics[height=1.7ex,keepaspectratio]{shared/logo-singularity.png}}
\newcommand{\logoSlurm}{\protect\includegraphics[height=1.7ex,keepaspectratio]{shared/logo-slurm.png}}
\newcommand{\logoSnakemake}{\protect\includegraphics[height=1.7ex,keepaspectratio]{shared/logo-snakemake.png}}
\newcommand{\logoVirtualbox}{\protect\includegraphics[height=1.7ex,keepaspectratio]{shared/logo-virtualbox.png}}
\newcommand{\logoZenodo}{\protect\includegraphics[height=1.7ex,keepaspectratio]{shared/logo-zenodo.png}}


%The next statement creates the title page.
\frame{\titlepage}

%---------------------------------------------------------
% General information
\begin{frame}{General information}
\begin{columns}
\column{0.6\textwidth}
\begin{block}{Practical information:}
\begin{itemize}
    \item Dates: August 31st - September 2nd
    \item Location: Institut des Systèmes Complexes, 113 rue Nationale, 75013-Paris
    \item Courses: 9:00 to 17:30
    \item Meal: 12:30-14:00
    \item Pauses: 10:30-11:00 + 15:30-16:00
    \item 2 days of courses + 1 day of course building
\end{itemize}
\end{block}
\column{0.2\textwidth}
\begin{block}{Round table:}
    \begin{itemize}
        \item Teachers 
        \item Learners
    \end{itemize}
\end{block}
\begin{block}{Ressources:}
    \begin{itemize}
        \item \includegraphics[height=0.5cm]{shared/CC-by-nc-sa.png}
        \item GitLab
        \item \LaTeX
    \end{itemize}
\end{block}
\end{columns}
\end{frame}
%---------------------------------------------------------
% Schelude
% https://docs.google.com/document/d/11ApQxig5JAvkAsYKIbBR0SvlojPeuqoNM34g1r6P__Q/
\begin{frame}
\frametitle{Training schedule}
Day 1:
\begin{itemize}
    \item \hyperlink{Introduction}{Introduction to \FAIRB}
    \item \hyperlink{Encapsulation}{Encapsulation (2 Practical Sessions, \includegraphics[height=0.4cm]{shared/logo-docker-paysage.png})}
    \item \hyperlink{Workflow}{Workflow (2 PS, \includegraphics[height=0.4cm]{shared/logo-snakemake.png})}
    \item \hyperlink{IFB}{IFB resources (2 PS, \includegraphics[height=0.3cm]{shared/logo-singularity.png}, \includegraphics[height=0.3cm]{shared/logo-slurm.png})}
\end{itemize}
Day 2:
\begin{itemize}
    \item \hyperlink{History_management}{History management (3 PS, \includegraphics[height=0.3cm]{shared/logo-git.png}, \includegraphics[height=0.4cm]{shared/logo-github.png})}
    \item \hyperlink{Software_environment}{Software environment management (1 PS, \includegraphics[height=0.2cm]{shared/logo-conda.png})}
    \item \hyperlink{Notebooks}{Traceability with notebooks (2 PS, \includegraphics[height=0.3cm]{shared/logo-jupyter.png}, \includegraphics[height=0.3cm]{shared/logo-Rmarkdown.png})}
    \item \hyperlink{Sharing}{Sharing and dissemination (1 PS, \includegraphics[height=0.4cm]{shared/logo-zenodo.png})}
    \item \hyperlink{Conclusion}{Conclusion}
\end{itemize}
Day 3: 
\begin{itemize}
    \item \hyperlink{Improvement}{Empowerment and improvement of resources (2 PE)}
\end{itemize}
\end{frame}
%---------------------------------------------------------
%This block of code is for the table of contents 
%\begin{frame}
%\frametitle{Table of Contents}
%\tableofcontents
%\end{frame}
%---------------------------------------------------------
%---------------------------------------------------------
% Introduction
\label{Introduction}
\section[Introduction]{Introduction to \FAIRB}
\input 01_introduction/01_introduction
%---------------------------------------------------------
% Encapsulation
\label{Encapsulation}
\section[Encapsulation]{Encapsulation}
\input 02_encapsulation/02_encapsulation
%---------------------------------------------------------
% Workflow
\label{Workflow}
\section[Pipeline]{Workflow}
\input 03_workflow/03_workflow
%---------------------------------------------------------
% IFB resources
\label{IFB}
\section[IFB]{IFB resources}
\input 04_IFB/04_IFB
%---------------------------------------------------------
% History management
\label{History}
\section[History]{History management}
\input 05_history/05_history
%---------------------------------------------------------
% Environment
\label{Software_Environment}
\section[Environment]{Software Environment}
\input 06_environment/06_environment
%---------------------------------------------------------
% Notebook
\label{Notebooks}
\section{Tracability with Notebook}
\input 07_notebook/07_notebook
%---------------------------------------------------------
% Sharing
\label{Sharing}
\section{Sharing and dissemination}
\input 08_sharing/08_sharing
%---------------------------------------------------------
% Conclusion
\label{Conclusion}
\section{Conclusion}
\input 09_conclusion/09_conclusion
%---------------------------------------------------------

\section{First section}

%---------------------------------------------------------
%Changing visivility of the text
\begin{frame}
\frametitle{Sample frame title}
This is a text in second frame. For the sake of showing an example.

\begin{itemize}
    \item<1-> Text visible on slide 1
    \item<2-> Text visible on slide 2
    \item<3> Text visible on slides 3
    \item<4-> Text visible on slide 4
\end{itemize}
\end{frame}

%---------------------------------------------------------


%---------------------------------------------------------
%Example of the \pause command
\begin{frame}
In this slide \pause

the text will be partially visible \pause

And finally everything will be there
\end{frame}
%---------------------------------------------------------

\section{Second section}

%---------------------------------------------------------
%Highlighting text
\begin{frame}
\frametitle{Sample frame title}

In this slide, some important text will be
\alert{highlighted} because it's important.
Please, don't abuse it.

\begin{block}{Remark}
Sample text
\end{block}

\begin{alertblock}{Important theorem}
Sample text in red box
\end{alertblock}

\begin{examples}
Sample text in green box. The title of the block is ``Examples".
\end{examples}
\end{frame}
%---------------------------------------------------------


%---------------------------------------------------------
%Two columns
\begin{frame}
\frametitle{Two-column slide}

\begin{columns}

\column{0.5\textwidth}
This is a text in first column.
$$E=mc^2$$
\begin{itemize}
\item First item
\item Second item
\end{itemize}

\column{0.5\textwidth}
This text will be in the second column
and on a second tought this is a nice looking
layout in some cases.
\end{columns}
\end{frame}
%---------------------------------------------------------


\end{document}