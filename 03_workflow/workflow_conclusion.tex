%-------------------------------------------
\begin{frame}[containsverbatim]
\frametitle{Last challenge}
%-------------------------------------------
\begin{exampleblock}{Clean, delete and re-run !}
We may saved the last version of the snakefile and the config file, clean all (but the data) and re-run the workflow.
\begin{lstlisting}
cp ex?_o?.smk RNAseq_analysis.smk
cp ex?_o?.yml RNAseq_analysis_smkEnv.yml
rm -Rf FastQC/ Results/ Logs/ Tmp/ multiqc_*
snakemake -c 1 -s  RNAseq_analysis.smk --configfile RNAseq_analysis_smkEnv.yml
\end{lstlisting}
\end{exampleblock}
\end{frame}
%-------------------------------------------
\begin{frame}{Snakemake conclusion}
%-------------------------------------------
Now you can transpose/write any shell script to a snakefile and associate it to a configuration file.
\begin{block}{Power gain}
\begin{itemize}
    \item This 2-files solution (snake $\&$ config files) will be more powerful when you apply it in a High Performance Computing environment (like the IFB cluster) if you arrange to put all paths in the config file
    \item I tune up my snakefile with a reduced dataset (typically the first ~10000 reads of each input fastq file) before running the full analysis
    \item For analysis with many data files Snakemake handles error recovery from unintentional interruptions for us: just rerun the snakemake command until each file is processed
\end{itemize}
\end{block}
\begin{block}{Reprodicibility issue}
In terms of reproducibility, we have to focus on the tools environment
\end{block}
\end{frame}
%-------------------------------------------
\begin{frame}{Ressources}
\begin{description}
    \item [Official documentation] https://snakemake.readthedocs.io/en/stable/
    \item [Johannes Koëster publication] https://doi.org/10.1093/bioinformatics/bts480
    \item [bioinfo-fr.net]  https://bioinfo-fr.net (+search snakemake)
    \item [begining of a gitbook] https://endrebak.gitbooks.io/the-snakemake-book
\end{description}
    
\end{frame}
