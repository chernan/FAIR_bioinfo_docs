%-------------------------------------------
\begin{frame}[containsverbatim]
\frametitle{Conda setup}
%-------------------------------------------
\begin{exampleblock}{How to access conda?}
\begin{itemize}
    \item Conda is so used that it could even be installed by default to your machine. To test this: \verb|conda --version|
    \item if not, may install it or got it by a docker image:
\begin{lstlisting}
docker run -i -t -v ${PWD}:/data continuumio/miniconda3
\end{lstlisting}
   \item already activated on the IFB cluster (otherwise with module: \verb|module load conda|)
\end{itemize} 
\end{exampleblock}
%\begin{exampleblock}{Conda environment}
%We have already (blindly) use a conda configuration file in the workflow session:
%\begin{lstlisting}
%conda env create -n envfair -f envfair.yml
%conda activate envfair
%\end{lstlisting}
%We will next detail the content of the configuration file, \verb|envfair.yml|
%\end{exampleblock}
\end{frame}

%-------------------------------------------
\begin{frame}[containsverbatim]
\frametitle{How to access tools?}
%-------------------------------------------
\begin{exampleblock}{Manage Conda environment}
\begin{enumerate}
    \item create the working environment:
\begin{lstlisting}
conda create env -n myenv
\end{lstlisting}
    \item activate it: 
\begin{lstlisting}
conda activate myenv
\end{lstlisting}
    \item if not yet done, install packages (specify the channel): 
\begin{lstlisting}
conda install -c bioconda bowtie2
\end{lstlisting}
    \item work with the tools
    \item quite the environment: 
\begin{lstlisting}
conda deactivate
\end{lstlisting}
\end{enumerate}
\end{exampleblock}
\end{frame}
%-------------------------------------------
\begin{frame}[containsverbatim]
\frametitle{Install snakemake with conda}
%-------------------------------------------
\begin{exampleblock}{Objective}
Create a conda configuration file to install the snakemake tool.
\end{exampleblock}
\begin{exampleblock}{Hint}
\begin{itemize}
    \item Search its channel in the Anaconda cloud web pages
    \item the "minimal" environment is sufficient
\end{itemize}
\end{exampleblock}
\end{frame}
%-------------------------------------------
\begin{frame}[containsverbatim]
\frametitle{Install snakemake with conda}
%-------------------------------------------
\begin{exampleblock}{condaEnvSmk.yml}
\begin{lstlisting}
channels:
  - conda-forge
  - bioconda
  - main
dependencies:
  - snakemake-minimal=6.5.0
\end{lstlisting}
\end{exampleblock}
\begin{exampleblock}{run}
\begin{lstlisting}
conda env create -n condaEnvSmk -f condaEnvSmk.yml
conda activate condaEnvSmk
snakemake --version
\end{lstlisting}
\end{exampleblock}
\end{frame}
%-------------------------------------------
%\begin{frame}[containsverbatim]
%\frametitle{Example of a conda configuration file}
%-------------------------------------------
%\begin{exampleblock}{envfair.yml}
%\begin{lstlisting}[language=python]
%channels:
%  - conda-forge
%  - bioconda
%  - main
%  - default
%dependencies:
%  - python=3.7.6 # specify python version (not required but can help with downstream conflicts)
%  - snakemake-minimal=5.10.0 # workflow manager
%  - graphviz=2.42.3 # for visualisation
%  - xorg-libxrender
%  - xorg-libxpm
%  - wget=1.20.1 # for downloading files
%  - fastqc=0.11.9 # for the RNAseq analysis
%  - bowtie2=2.4.1
%  - samtools=1.10
%  - subread=2.0.1
%\end{lstlisting}
%\end{exampleblock}
%\end{frame}