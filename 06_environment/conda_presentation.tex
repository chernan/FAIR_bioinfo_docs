% \begin{frame}{Extrait du programme IFB}
% Pourquoi un gestionnaire d’environnements et de packages ?\\
% Installation de packages avec Conda\\
% Utilisation d’environnement Conda\\
%
%TP: Création d’un script d’installation avec conda :\\
% installer snakemake avec conda\\
% \end{frame}
%----------------------------------------
\begin{frame}{\includegraphics[height=0.5cm]{shared/logo-conda.png}: an environment manager}
%----------------------------------------
\begin{block}{Conda definitions}
    \begin{itemize}
        \item Environment: a set of packages/tools in a directory (added to our PATH)
        \item Conda: an open source package + a general-purpose environment management system (installation, execution, upgrade). For any programming language, multi-platform (Windows, MacOS, Linux).
        \item Conda package: a compressed tarball of a tool
    \end{itemize}
\end{block}
\begin{block}{Why using an environment manager?}
\begin{itemize}
    \item avoid compilation and dependencies problems: an environment manager will take care of everything!
    \item have several environments in parallel each with their own set of tools
    \item useful when cross-tools dependencies are incompatible with each other
\end{itemize}
\end{block}
\end{frame}
%----------------------------------------
\begin{frame}{\includegraphics[height=0.5cm]{shared/logo-conda.png}: Access}
%----------------------------------------
\begin{block}{Conda distribution}
\begin{itemize}
        \item Anaconda: a data science platform, comes with a lot of packages
        \item Miniconda: come without installed packages
\end{itemize}
\end{block}
\begin{block}{Anconda cloud, the "conda hub"}
\begin{itemize}
    \item \href{http://anaconda.org/}{\textcolor{blue}{\underline{Anaconda cloud}}}
    (private company) relies on the community of developers, concerns many domains (Machine Learning, Data Visualization, Dashboarding-web, Image Processing, Natural Language Processing, etc)
    \item Anaconda cloud: made up of channels/owners. Each channels contains one or more conda packages 
    \item be careful when downloading any packages from an untrusted source, always inspect before installation
\end{itemize}
\end{block}
\end{frame}
%----------------------------------------
\begin{frame}[containsverbatim]
{\includegraphics[height=0.5cm]{shared/logo-conda.png} About channels}
%----------------------------------------
\begin{block}{Some conda channels}
\begin{itemize}
    \item \verb|default|
    \item \verb|conda-forge|: many popular python packages (analogous to PyPI but with a unified, automated build infrastructure and more peer review of recipes)
    \item \verb|bioconda|: bioinformaticians' contributions
    \item private
\end{itemize}
\end{block}

\begin{block}{Channels list order}
\begin{itemize}
    \item when different channels have the same package $\Rightarrow$ collisions
    \item collisions resolved following the order of your channels list $\Rightarrow$ put supplemental channels at the bottom of your channel list
\end{itemize}
\end{block}
\end{frame}
%----------------------------------------
\begin{frame}[containsverbatim]
{\includegraphics[height=0.5cm]{shared/logo-conda.png} R, mamba}
%----------------------------------------
\begin{block}{Conda and R}
    The R interpreter is included in the \verb|r-essentials| package (~200 r packages).
    Add \verb|r-| before the regular r package name (eg. \verb|r-ggplot2|)
    \begin{center}
        \includegraphics[height=3cm]{06_environment/Images/conda_Ressentials.png}
    \end{center}
\end{block}
\begin{block}{Mamba}
A fast drop-in alternative to conda, using libsolv for dependency resolution
\begin{lstlisting}[language=python]
conda install -c conda-forge mamba
\end{lstlisting}
Next, replace \verb|conda| by \verb|mamba| to use it
\end{block}
\end{frame}
%----------------------------------------
\begin{frame}[containsverbatim]
\frametitle{\includegraphics[height=0.5cm]{shared/logo-conda.png} command}
%----------------------------------------
\begin{block}{simple commands}
\begin{lstlisting}[language=python]
conda create env -n myenv # creation of a conda environment 
conda info --envs # list environments (* for the active one) 
conda activate myenv # active the myenv environment
conda list # list packages (only in an active environment)
conda install package # installation of a tool/package
conda remove package # suppress the tool from the environment
conda env remove -n myenv # suppress the myenv environment
conda deactivate # inactivate the environment
\end{lstlisting}
\end{block}
\begin{block}{miniconda3}
With the miniconda3 distribution and by default, environments are installed in a \verb|miniconda3/envs/| repository
\end{block}
\end{frame}
%----------------------------------------
\begin{frame}[containsverbatim]
\frametitle{\includegraphics[height=0.5cm]{shared/logo-conda.png} 2 modes}
%----------------------------------------
\begin{block}{interactive}
\begin{itemize}
    \item create an environment
    \item activate the environment
    \item install some conda packages
\end{itemize}
\end{block}
\begin{block}{configuration file}
\begin{itemize}
    \item list all conda packages in a configuration file (\verb|yaml| or \verb|json| format)
    \item create the environment based on the configuration file (option \verb|-f|)
    \item activate the environment
\end{itemize}
\end{block}
\begin{block}{reproducibility}
\begin{itemize}
    \item good practice: use a configuration file
    \item specify a precise version of a package:  \verb|<channel>::<package>=<version>| 
\end{itemize}
\end{block}
\end{frame}